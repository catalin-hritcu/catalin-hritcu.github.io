\documentclass[a4paper,11pt]{article}

%% ** Packages
\usepackage{fullpage}
\usepackage{enumitem}

\usepackage{afterpage}
\usepackage{color}
\usepackage{amssymb,amsmath,amsfonts}
\usepackage{extarrows}
\usepackage{mathptmx}
\usepackage{version}
\usepackage{xspace}
\usepackage{graphicx}
\usepackage{listings}
\usepackage{wrapfig}
\usepackage{ifpdf}
\usepackage{semantic}
\usepackage{natbib}
\newcommand\citepos[1]{\citeauthor{#1}'s\ \citeyear{#1}}
\bibpunct();A{},
\let\cite=\citep
\usepackage{amsthm}
\usepackage{stmaryrd}
\usepackage{subfigure}
\usepackage{multirow}
\usepackage{proof}
\usepackage{mathpartir}
\definecolor{darkblue}{rgb}{0.0,0.0,0.3}

\usepackage{hyperref}
\lstloadlanguages{caml}

% \include{preamble}
% Something in preamble.tex causes a hard failure
%   Undefined control sequence.
%   <argument> \mdseries@rm
% using minimal preamble instead
\newcommand\maybecolor[1]{\color{#1}}

\newcommand\mypara[1]{\vskip 0.03in \noindent{\textbf{\itshape{#1}}}\quad}

\newcommand{\aset}[1]{{\ensuremath{\{#1\}}}}

\def\lstlanguagefiles{lstfstar.tex}
\lstset{language=fstar}
\let\ls\lstinline

\hyphenation{meta-language}

\newcommand\fstar{F$^\star$\xspace}
\newcommand\microfstar{$\mu$\fstar} % EMF* emf*
\newcommand\picofstar{$p$\fstar}

\newcommand\emf{{\sc emf}$^\star$\xspace}
\newcommand\deflang{{\sc dm}\xspace} %dm, evoking Dijkstra monad, defining monads, deriving monads, etc.
\newcommand\emfST{{\sc emf}$^\star_{\text {\sc st}}$\xspace} % consistent with 2 others

\newcommand{\un}[1]{\ensuremath{\underline{#1}}}
\newcommand{\F}[2]{\mathrm{F}_{#1}~#2}
\newcommand{\G}[3]{\mathrm{G}^{#1}_{#2}(#3)}
\newcommand{\returnT}[1]{\mathrm{\bf return}_{\teff}~#1}
\newcommand{\bindT}[3]{\mathrm{\bf bind}_{\teff}~#1~\mathrm{\bf to}~#2~\mathrm{\bf in}~#3}

\newcommand{\sub}{\mathrm{s}}
\newcommand{\sG}{\sub_\Gamma}

\newcommand{\neu}{\ensuremath{n}}
\newcommand{\teff}{\ensuremath{\tau}}
\newcommand{\narr}{\xrightarrow{\neu}}
\newcommand{\tarr}{\xrightarrow{\teff}}
\newcommand{\earr}{\xrightarrow{\epsilon}}
\newcommand{\dneg}[1]{(#1 -> \typez) -> \typez}
\newcommand{\DmG}{\Delta\mid\Gamma}
\newcommand{\fst}[1]{\mathrm{\bf fst}(#1)}
\newcommand{\snd}[1]{\mathrm{\bf snd}(#1)}
\newcommand{\inl}[1]{\mathrm{\bf inl}(#1)}
\newcommand{\inr}[1]{\mathrm{\bf inr}(#1)}
\newcommand{\mycases}[3]{\mathrm{\bf case}~#1~\mathrm{\bf inl}~#2;~\mathrm{\bf inr}~#3}
\newcommand{\eqdef}{=_{\mathrm{\bf def}}}
\newcommand{\bang}{\,!\,}

\newcommand{\cps}[1]{\ensuremath{#1^{\star}}}          % notation for a CPS'd term
\newcommand{\cpst}[0]{$\star$-translation\xspace}  % name of the CPS transform
\newcommand{\cpsts}[0]{$\star$-translations\xspace}  % name of the CPS transform (plural)

% "Stronger than"
\newcommand{\stg}{\mathrel{\lesssim}}

% Colors
\definecolor{dkblue}{rgb}{0,0.1,0.5}
\definecolor{dkgreen}{rgb}{0,0.4,0}
\definecolor{dkred}{rgb}{0.6,0,0}
\definecolor{dkpurple}{rgb}{0.7,0,1.0}
\definecolor{purple}{rgb}{0.9,0,1.0}
\definecolor{olive}{rgb}{0.4, 0.4, 0.0}
\definecolor{teal}{rgb}{0.0,0.4,0.4}
\definecolor{azure}{rgb}{0.0, 0.5, 1.0}

% Comments
\newcommand{\comm}[3]{\ifcheckpagebudget\else\ifdraft{\maybecolor{#1}[#2: #3]}\fi\fi}
\newcommand{\nik}[1]{\comm{dkpurple}{Nik}{#1}}
\newcommand{\ch}[1]{\comm{teal}{CH}{#1}}
\newcommand{\cf}[1]{\comm{teal}{CF}{#1}}
\newcommand{\chfoot}[1]{\ifdraft\footnote{\comm{teal}{CH}{#1}}\fi}
%\newcommand{\gm}[1]{\comm{dkgreen}{GM}{#1}} % Guido
\newcommand{\nig}[1]{\comm{dkgreen}{Niklas}{#1}}
\newcommand{\gdp}[1]{\comm{blue}{GDP}{#1}} % Gordon
\newcommand{\aseem}[1]{\comm{magenta}{Aseem}{#1}}
\newcommand{\jp}[1]{\comm{olive}{JP}{#1}}
\newcommand{\km}[1]{\comm{purple}{KM}{#1}}

\newcommand*{\EG}{e.g.,\xspace}
\newcommand*{\IE}{i.e.,\xspace}
\newcommand*{\ETAL}{et al.\xspace}
\newcommand*{\ETC}{etc.\xspace}

\newcommand\surl[1]{{\small\url{#1}}}
\newcommand\ssurl[1]{{\scriptsize\url{#1}}}

\begin{document}

\author{}

\title{MPRI 2:30: Semester II Exam (\fstar{} Part)}

\date{}

\maketitle

\paragraph{Question 1 (Abstract queue)}
Consider the following module implementing (FIFO) queues using lists,
but keeping this representation of queues and all the operations abstract.
\begin{lstlisting}
module AbstractQueue
  abstract type queue = list int
  abstract val is_empty : queue -> bool
  let is_empty = Nil?
  abstract val empty : q:queue{is_empty q}
  let empty = []
  abstract val enq : int -> queue -> q:queue{~(is_empty q)}
  let enq x xs = Cons x xs
  abstract val deq : q:queue{~(is_empty q)} -> queue
  let rec deq xs = match xs with
                   | [x] -> []
                   | x1::x2::xs -> x1 :: deq (x2::xs)
  abstract val front : q:queue{~(is_empty q)} -> int
  let rec front xs = match xs with
                     | [x] -> x
                     | x1::x2::xs -> front (x2::xs)
\end{lstlisting}
To allow client code to use this interface and reason about the
results we export lemmas providing an algebraic specification of
queues. For instance, the following \ls$front_enq$ lemma
\begin{lstlisting}
  let front_enq (i:int) (q:queue) : Lemma (front (enq i q) = (if is_empty q then i else front q)) = ()
\end{lstlisting}
allows us to prove the following assertion {\em outside} the
\ls$AbstractQueue$ module:
\begin{lstlisting}
assert(front (enq 3 (enq 2 (enq 1 empty))) = 1)
\end{lstlisting}
\begin{enumerate}[label={\bf (\alph*)},topsep=0pt]
\item How many instances of the \ls$front_enq$ lemma are needed to show
      the assert above? Write down these instances.
\vspace{2.5cm}
\item Write another lemma that can be exposed by the
  \ls$AbstractQueue$ module to prove the assertion:
\begin{lstlisting}
assert(deq (enq 3 (enq 2 (enq 1 empty))) == enq 3 (enq 2 empty))
\end{lstlisting}
\end{enumerate}

\clearpage

\paragraph{Question 2 (Imperative count)}

Write down a valid specification for the following stateful function:
\begin{lstlisting}
let rec count r (n:nat) : ST unit (requires (fun h0 -> ............................................................................))

                             (ensures (fun h0 () h1 -> .....................................................................)) = 
  if n > 0 then (r := !r + 1; count r (n $-$ 1))
\end{lstlisting}
Your specification should capture what this function does in terms of
the usual \ls$sel$ and \ls$upd$ specification-level functions:
\begin{lstlisting}
val sel : #a:Type -> heap -> ref a -> GTot a
val upd : #a:Type -> heap -> ref a -> a -> GTot heap
\end{lstlisting}
and can assume the usual theory of arrays for \ls$sel$ and \ls$upd$.


\paragraph{Question 3 (Removing from lists)}

The following \fstar{} function removes all occurrences of
an integer \ls$y$ from a list \ls$xs$:
\begin{lstlisting}
let rec remove (y:int) (xs:list int) : Tot (list int) (decreases xs) =
  match xs with
  | [] -> []
  | x :: xs' -> if x = y then remove x xs' else x :: remove y xs'
\end{lstlisting}
%
We can use \ls$remove$ to implement another function that removes all
elements of a list \ls$ys$ from a list \ls$xs$:
\begin{lstlisting}
let rec remove_list (ys:list int) (xs:list int) : Tot (list int) (decreases ys) =
  match ys with
  | [] -> xs
  | y'::ys' -> remove y' (remove_list ys' xs)
\end{lstlisting}
%
You will have to prove that \ls$remove_list$ indeed removes all the
elements of \ls$ys$ from \ls$xs$. More precisely, given a
function that counts all occurrences of an element \ls$y$ in a list \ls$xs$:
%
\begin{lstlisting}
let rec count (y:int) (xs:list int) : Tot nat (decreases xs) =
  match xs with
  | [] -> 0
  | x :: xs' -> (if x = y then 1 else 0) + count y xs'
\end{lstlisting}
%
prove the following main lemma:
\begin{lstlisting}
val count_remove_list (y:int) (ys:list int) (xs:list int) :
    Lemma (requires (count y ys > 0)) (ensures (count y (remove_list ys xs) = 0))
\end{lstlisting}
%
Write down any intermediate lemmas you use in your proof in \fstar{}
syntax and prove those as well.
%
The proofs of the main and intermediate lemmas can be in whatever
syntax you like (\fstar{}, math, a mix) and for each of these proofs
make it explicit over what you are doing induction, what is the case
structure, where you are using other lemmas, etc.

\end{document}
